\documentclass[letterpaper,11pt]{article}

\usepackage{latexsym}
\usepackage[empty]{fullpage}
\usepackage{titlesec}
\usepackage{marvosym}
\usepackage[usenames,dvipsnames]{color}
\usepackage{verbatim}
\usepackage{enumitem}
\usepackage[hidelinks]{hyperref}
\usepackage{fancyhdr}
\usepackage[english]{babel}
\usepackage{tabularx}
\usepackage{fontawesome5}
\input{glyphtounicode}

\pagestyle{fancy}
\fancyhf{} % clear all header and footer fields
\fancyfoot{}
\renewcommand{\headrulewidth}{0pt}
\renewcommand{\footrulewidth}{0pt}

% Adjust margins
\addtolength{\oddsidemargin}{-0.5in}
\addtolength{\evensidemargin}{-0.5in}
\addtolength{\textwidth}{1in}
\addtolength{\topmargin}{-.5in}
\addtolength{\textheight}{1.0in}

\urlstyle{same}

\raggedbottom
\raggedright
\setlength{\tabcolsep}{0in}

% Sections formatting
\titleformat{\section}{
  \vspace{-8pt}\scshape\raggedright\large
}{}{0em}{}[\color{black}\titlerule \vspace{-3pt}]

% Ensure that generate pdf is machine readable/ATS parsable
\pdfgentounicode=1

%-------------------------
% Custom commands
\newcommand{\resumeItem}[2]{
  \item\small{
    \textbf{#1}{: #2 \vspace{-2pt}}
  }
}

%% ==== Headings ==== 
% Just in case someone needs a heading that does not need to be in a list
\newcommand{\resumeEducationHeading}[5]{
    \begin{tabular*}{0.99\textwidth}[t]{l@{\extracolsep{\fill}}r}
      \textbf{#1} $\vert$ #2 & #3 \\
      \textit{\small#4} & \textit{\small #5} \\
    \end{tabular*}\vspace{0.5pt}
}

\newcommand{\resumeExperienceHeading}[3]{
    \begin{tabular*}{0.99\textwidth}[t]{l@{\extracolsep{\fill}}r}
      \textbf{#1} $\vert$ {#2} & {#3} \\
    \end{tabular*}\vspace{-3pt}
}

\newcommand{\resumeProjectHeading}[3]{
    \begin{tabular*}{0.97\textwidth}[t]{l@{\extracolsep{\fill}}r}
      \textbf{#1} $\vert$ \textit{#2}  & {#3}
    \end{tabular*}\vspace{-3pt}
}

%% ==== Subheadings ====
\newcommand{\resumeSubheading}[4]{
  \vspace{-1pt}\item
    \begin{tabular*}{0.97\textwidth}[t]{l@{\extracolsep{\fill}}r}
      \textbf{#1} & #2 \\
      \textit{\small#3} & \textit{\small #4} \\
    \end{tabular*}\vspace{-5pt}
}

\newcommand{\resumeSubSubheading}[2]{
    \begin{tabular*}{0.97\textwidth}{l@{\extracolsep{\fill}}r}
      \textit{\small#1} & \textit{\small #2} \\
    \end{tabular*}\vspace{-5pt}
}

\newcommand{\resumeSubItem}[2]{\resumeItem{#1}{#2}\vspace{-4pt}}

\renewcommand{\labelitemii}{$\circ$}

\newcommand{\resumeSubHeadingListStart}{\begin{itemize}[leftmargin=*]}
\newcommand{\resumeSubHeadingListEnd}{\end{itemize}}
\newcommand{\resumeItemListStart}{\begin{itemize}[noitemsep]\vspace{-4pt}}
\newcommand{\resumeItemListEnd}{\end{itemize}}


%-------------------------------------------
%%%%%%  Resume STARTS HERE  %%%%%%%%%%%%%%%%%%%%%%%%%%%%

\begin{document}

%----------HEADING-----------------
\begin{center}
  \textbf{\huge Brian Pak} \\
  \vspace*{0.1cm}
  {bpak7@gatech.edu} $\cdot$ {brianpak.me} $\cdot$ {github.com/brianpak2402} $\cdot$ {linkedin.com/in/brianpakk}  $\cdot$ Atlanta, GA
\end{center}

%-----------EDUCATION-----------------
\section{Education}
    \resumeEducationHeading
      {Georgia Institute of Technology}{Atlanta, GA}{Expected: May 2024}
      {B.S -- Computer Science, Concentrations: Intelligence \& Media}{\vspace{0.1cm}} 
    \textbf{Coursework}{: Data Structures \& Algorithms (Java \& C++), Software Design, Information Visualization, Computer Organization \& Programming, Object-Oriented Programming, Discrete Math, Combinatorics} \\

%--------------EXPERIENCE---------------
% \section{Experience}
%   \resumeExperienceHeading{GT WebDev}{Student Software Developer}{August 2022 - Present}
%     \resumeItemListStart
%       \item {Spearheaded the implementation of the Authentication Flow with PKCE Extension using \textbf{Axios} and \textbf{AWS Lambda}, allowing a user-specific access token to be securely stored in \textbf{DynamoDB} for later use.}
%       \item {Designed the Authentication Flow and Search for Songs functionality by composing two Technical Design Documents, informing other team members about its uses and potential challenges.}
%       \item {Maintained real-time data of `up-votes' and `down-votes' of certain songs/albums using \textbf{WebSockets API}, allowing many people to interact with a music session simultaneously.}
%     \resumeItemListEnd
  
%-----------SKILLS----------------------
\section{Skills}
    \textbf{Languages}{: Java, JavaScript, HTML/CSS, Typescript, SQL, C, Python, C++} \\
    \textbf{Technologies}{: Git, React, Next.js,  DynamoDB, AWS Lambda, Axios, Serverless Stack Toolkit (SST), Tailwind CSS, Spring Boot, REST APIs}

%-----------PROJECT EXPERIENCE-----------------
\section{Projects}
    \resumeProjectHeading{Spotify Jukebox}{SST, Axios, Chakra UI, WebSockets API, DynamoDB}{August 2022 - Present}
      \resumeItemListStart
        \item {Spearheaded the implementation of seven RESTful API endpoints using \textbf{Axios} and \textbf{AWS Lambda}, providing users with the ability to play songs through \textbf{Spotify API}.}
        \item {Implemented the \textbf{OAuth} Authorization Code Flow with PKCE to securely verify a host user's Spotify account before beginning a music session.}
        \item {Maintained real-time data of `up-votes' and `down-votes' of jukebox song queues with\textbf{WebSockets API}.}
        \item {Redesigned the user interface for the local song queues using \textbf{Chakra UI}, providing users with an improved experience with adding/removing songs from the jukebox queue.}
      \resumeItemListEnd

    \resumeProjectHeading{BuzzConnect}{Spring Boot, React, SQL}{August 2022 - December 2022}
      \resumeItemListStart
        \item {Developed a full-stack \textbf{React} application in an Agile devleopment process to assist students in exploring current student events on Georgia Tech's campus.}
        \item {Headed the development of a RESTful API with over 15 endpoints using \textbf{Spring Boot}, allowing successful transfer of user and event data.}
        \item {Modeled relationships between four different functional classes using \textbf{Spring Boot JPA}, allowing the development team to easily maintain and store data in a local \textbf{MySQL} database.}
      \resumeItemListEnd

    \resumeProjectHeading{Unix Utilities}{C}{June 2021 - August 2021}
      \resumeItemListStart
        \item {Built a collection of 8 \textbf{Linux/UNIX} command-line commands, including head, tail, env, and wc.}
        \item {Employed \textbf{C Standard Library} Functions to make system calls to the Linux Kernel, allowing the implementation work directly with the device's operating system.}
      \resumeItemListEnd

    \resumeProjectHeading{Stranger Things Showdown}{C}{June 2022 - August 2022}
      \resumeItemListStart
        \item {Programmed a Stranger-Things-themed rock, paper, scissors video game for the Gameboy Advance, hosted on an emulator provided through a Docker container.}
        \item {Encoded graphics for 7 protagonists and background scenery by manipulating direct memory access functions provided with \textbf{C programming}.}
      \resumeItemListEnd

%-----------LEADERSHIP----------------------
\section{Extracurriculars}
  \resumeExperienceHeading{GT WebDev Club}{Co-Lead \& Student Developer}{August 2022 - Present}
    \resumeItemListStart
      \item {Migrated the application's user interface to \textbf{Next.js}, \textbf{NextUI}, \& \textbf{Vanilla Extract UI} to take advantage of sever-side rendering when rendering updates on a user's local song queue.}
      \item {Directed the backend and API development teams in expanding the application's API to support financial transactions with \textbf{Stripe API}.}
      \item {Organized 6 sprints for the development by assigning Github Issues into sprints to be completed by members, maintaining a steady pace for the development of the project.}
      \item {Developed over 12 unit tests using \textbf{Vitest} to ensure that API endpoints and Lambda functions are executing as intended.}
    \resumeItemListEnd


\end{document}